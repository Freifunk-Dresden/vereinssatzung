\documentclass[parskip]{scrartcl}
\usepackage[juratotoc, clausemark=forceboth]{scrjura}
%\usepackage[utf8]{inputenc}
\usepackage[ngerman]{babel}
\usepackage{lmodern}
%\usepackage[T1]{fontenc}

\usepackage[headsepline]{scrpage2}
\pagestyle{scrheadings}
%\automark[subsection]{section}
\ihead[]{\headmark}
\chead[]{}
\ohead[]{\pagemark}
\cfoot[]{}

\usepackage{graphicx}

\usepackage{listings}

\usepackage{hyperref}
\hypersetup{
	pdftitle=Satzung,
    pdfsubject=Freifunk Dresden e. V.,
    pdfauthor={Diego Jahn, Mirko Tomaschek, Michael Bauschke, Sven Kinne, Torsten Rudolph},
    pdfcreator=Overleaf v2,
    linktoc=all,
    hidelinks
}

%\titlehead{Kopf}
\subject{Freifunk Dresden e. V.}
\title{Satzung}
\subtitle{2. Fassung\\\href{https://www.freifunk-dresden.de}{\includegraphics{assets/Logo_Freifunk_Dresden}}}
%\author{Diego Jahn \\ Mirko Tomaschek \\ Sven Kinne \\ Torsten Rudolph}
\date{geändert mit Beschluss der Mitgliederversammlung vom\\28. November 2018}
\publishers{Rosenwerk, Jagdweg 1-3, 01159 Dresden}

\begin{document}

\maketitle
\tableofcontents\newpage

\begin{contract} % https://komascript.de/node/1404
\Clause{title={Name, Sitz, Geschäftsjahr}}%, number=1}

Der Verein führt den Namen „Freifunk Dresden”. Sollte er in das Vereinsregister eingetragen werden, führt er anschließend den Zusatz e.~V. Danach lautet der Name des Vereins „Freifunk Dresden e.~V.“

Der Verein hat seinen Sitz in Dresden.

Das Geschäftsjahr des Vereins ist das Kalenderjahr.

\Clause{title={Zweck des Vereins; Gemeinnützigkeit}}

Zweck des Vereins ist die Bildung, Förderung und Weiterentwicklung der Verbreitung des Wissens über kabellose und kabelgebundene Computernetzwerke, die der Allgemeinheit zugänglich sind (freie Netzwerke) sowie dafür notwendige Grundlagen (Betriebssysteme, Netzwerke, ...) und Wissen zu dessen Anwendung (Routing und Netzwerkprotokolle, 802.11x, Meshverfahren, VPN, Webservices sowie andere Verfahren zum Betrieb einer solchen Infrastruktur).

Der Verein verfolgt ausschließlich und unmittelbar gemeinnützige Zwecke im Sinne des Abschnitts „Steuerbegünstigte Zwecke” der Abgabenordnung.

Der Satzungszweck wird verwirklicht insbesondere durch folgende Maßnahmen:

\begin{enumerate}
\item Information der Mitglieder, der Öffentlichkeit und interessierter Kreise über freie Netzwerke, insbesondere durch Vorträge, öffentliche Veranstaltungen und Vorführungen sowie Publikationen. Dies soll vor allem durch regelmäßige Sprechstunden für Interessierte Bürger erfolgen, um so einen zentralen Anlaufpunkt für eine Fortbildung zu ermöglichen.
\item Bereitstellung von Know-How über Technik und Anwendung freier Netzwerke in Form einer digitalen Bibliothek zum Ziele des Selbststudiums.
\item Bereitstellung von Information und Vermittlung von Wissen über gesellschaftliche, kulturelle, gesundheitliche, rechtliche und weitere Auswirkungen freier Netzwerke. Dies soll vor allem durch Veranstaltungen und Vorträge erfolgen.
\item Förderung der Kontakte und des Austauschs mit weiteren Personen und Organisationen im In- und Ausland, die im Bereich der freien Netzwerke tätig sind, vornehmlich zum Zwecke des Wissensaustausches und zur Weiterentwicklung.
\item Förderung und Unterstützung von Projekten und Initiativen, die in ähnlichen Bereichen tätig sind oder denen die Idee freier Netzwerke näher gebracht werden. Dies soll insbesondere durch Informationsveranstaltungen und Vorträge aber auch durch Schulungen von Mitgliedern der Organisationen erfolgen
\end{enumerate}

Der Verein ist selbstlos tätig, er verfolgt nicht in erster Linie eigenwirtschaftliche Zwecke.

Mittel des Vereins dürfen nur für die satzungsmäßigen Zwecke verwendet werden. Die Mitglieder erhalten keine Gewinnanteile und in ihrer Eigenschaft als Mitglieder auch keine sonstigen Zuwendungen aus Mitteln des Vereins. Es darf keine Person durch Ausgaben, die dem Zweck des Vereins fremd sind, oder durch unverhältnismäßig hohe Vergütungen begünstigt werden. Alle Inhaber von Vereinsämtern sind ehrenamtlich tätig.

Jeder Beschluss über die Änderung der Satzung ist vor dessen Anmeldung beim Registergericht dem zuständigen Finanzamt vorzulegen.

\Clause{title={Erwerb der Mitgliedschaft}}

Mitglieder des Vereins sind ordentliche Mitglieder, Ehrenmitglieder und Fördermitglieder. Ordentliche Mitglieder und Ehrenmitglieder sind aktiv und in der Mitgliederversammlung stimmberechtigt.

\begin{enumerate}
\item Ordentliche Mitglieder: Ordentliche Mitglieder sind nach eigenem Antrag vom Vorstand aufgenommene oder Gründungsmitglieder des Vereins. Sie sind stimm- und antragsberechtigt. Sie zahlen einen Beitrag gemäß Beitragsordnung.
\item Ehrenmitglieder sind nach Antrag durch die Mitgliederversammlung aufgenommene Mitglieder. Sie sind stimm- und antragsberechtigt. Sie zahlen gemäß Beitragsordnung keinen Beitrag.
\item Fördermitglieder sind nach eigenem Antrag durch den Vorstand bestätigte Mitglieder. Sie sind nicht stimm- oder antragsberechtigt. Sie zahlen einen freiwilligen Beitrag gemäß Beitragsordnung.
\end{enumerate}

Ordentliches Mitglied des Vereins kann jede Person werden, die sich mit den Zielen des Vereins verbunden fühlt und den Verein aktiv fördern will. Die Mitgliedschaft ist in elektronischer oder Textform (§ 126a BGB, 126b BGB) zu beantragen. Über den Antrag entscheidet der Vorstand. Der Antrag soll den Namen und die Anschrift des Antragstellers enthalten. Die Ablehnung eines Antrages auf Aufnahme bedarf keiner Begründung.

Fördermitglied des Vereins kann jede Person werden, die sich mit den Zielen des Vereins verbunden fühlt und den Verein finanziell und ideell unterstützen will. Die Mitgliedschaft ist in elektronischer oder Textform (§ 126a BGB, 126b BGB) zu beantragen. Über den Antrag entscheidet ein Vorstandsmitglied. Der Antrag soll den Namen und die Anschrift des Antragstellers enthalten. Die Ablehnung eines Antrages auf Aufnahme bedarf keiner Begründung.

Gegen den ablehnenden Bescheid des Vorstands kann der Antragsteller Beschwerde erheben. Die Beschwerde ist innerhalb eines Monats ab Zugang des ablehnenden Bescheids schriftlich per Einschreiben beim Vorstand einzulegen. Über die Beschwerde entscheidet die nächste ordentliche Mitgliederversammlung.

\Clause{title={Beendigung der Mitgliedschaft}}

Die Mitgliedschaft endet durch Austritt, Ausschluss oder Tod. Der Austritt aus dem Verein kann jederzeit schriftlich mit einer Frist von einem Monat gegenüber dem Vorstand erklärt werden. Beitragsrückstände eines Mitgliedes von mehr als einem Jahr gelten als Austrittserklärung.

Der freiwillige Austritt erfolgt gegenüber einem Mitglied des Vorstands in elektronischer oder Textform. Er ist nur zum Schluss eines Monats möglich.

Ein Mitglied kann, wenn es gegen die Vereinsinteressen gröblich verstoßen hat, durch Beschluss des Vorstands aus dem Verein ausgeschlossen werden. Vor der Beschlussfassung ist dem Mitglied unter Setzung einer angemessenen Frist Gelegenheit zu geben, sich persönlich vor dem Vorstand oder in Textform zu rechtfertigen. Eine in Textform vorliegende Stellungnahme des Betroffenen ist in der Vorstandssitzung zu verlesen. Der Beschluss über den Ausschluss ist mit Gründen zu versehen und dem Mitglied mittels eingeschriebenen Briefes bekannt zu machen. Gegen den Ausschließungsbeschluss des Vorstands steht dem Mitglied das Recht der Berufung an die Mitgliederversammlung zu. Die Berufung hat aufschiebende Wirkung. Die Berufung muss innerhalb einer Frist von einem Monat ab Zugang des Ausschließungsbeschlusses beim Vorstand schriftlich eingelegt werden. Ist die Berufung rechtzeitig eingelegt, so hat der Vorstand innerhalb von zwei Monaten die Mitgliederversammlung zur Entscheidung über die Berufung einzuberufen. Geschieht das nicht, gilt der Ausschließungsbeschluss als nicht erlassen. Macht das Mitglied von dem Recht der Berufung gegen den Ausschließungsbeschluss keinen Gebrauch oder versäumt es die Berufungsfrist, so unterwirft es sich damit dem Ausschließungsbeschluss mit der Folge, dass die Mitgliedschaft als beendet gilt.
Bei Adressverzug muss das Mitglied die neue Adresse angeben, ansonsten trägt das
Mitglied die durch die Ermittlung anfallenden Kosten selbst. 

\Clause{title={Mitgliedsbeiträge}}

Mitglieder entrichten einen jährlichen Beitrag, dessen Höhe von einer von der Mitgliederversammlung beschlossenen Beitragsordnung festgelegt ist.

Die Mittel für die Vereinszwecke sollen zusätzlich durch Zuwendungen, freiwillige Beiträge und durch Spenden aufgebracht werden.

Ehrenmitglieder sind von der Beitragspflicht befreit.

\Clause{title={Der Vorstand}}

Der Vorstand besteht aus mindestens drei (und höchstens fünf) Menschen. Die Mitgliederversammlung wählt grundsätzlich einen Vorsitzenden, einen Schatzmeister sowie einen stellvertretenden Vorsitzenden. Zusätzlich kann die Mitgliederversammlung so viele Beisitzer wählen, bis die Höchstanzahl von Mitgliedern erreicht ist. Der Vorstand kann für seine Tätigkeit eine Geschäftsordnung beschließen.

Für Rechtsgeschäfte ist die gemeinsame Vertretung durch zwei Vorstandsmitglieder erforderlich. Dabei muss eine der beiden Personen das Amt des Vorsitzenden oder des stellvertretenden Vorsitzenden begleiten.

Tritt ein Vorstandsmitglied zurück oder kann dieses seinen Aufgaben nicht mehr nachkommen, so geht seine Kompetenz, wenn möglich, auf ein anderes Vorstandsmitglied über. Der Vorstand gilt als nicht handlungsfähig, wenn ihm weniger als drei Vorstands-mitglieder angehören oder wenn der Vorstand sich selbst für handlungsunfähig erklärt. In einem solchen Fall ist schnellstmöglich eine außerordentliche Mitgliederversammlung einzuberufen und vom restlichen Vorstand zur Weiterführung der Geschäfte eine kommissarische Vertretung zu ernennen. Diese endet mit der Neuwahl des gesamten Vorstandes.

\Clause{title={Wahl und Amtsdauer des Vorstands}}

Der Vorstand wird von der Mitgliederversammlung auf die Dauer von zwei Jahren, vom Tage der Wahl an gerechnet, gewählt, er bleibt jedoch bis zur Neuwahl des Vorstands im Amt. Jedes Vorstandsmitglied ist einzeln zu wählen. Wählbar sind nur aktive Vereinsmitglieder.

Der Vorstand führt die Geschäfte des Vereins und verwaltet sein Vermögen. Der Vorstand ist verantwortlich für die ordnungsgemäße Verwaltung aller Ämter und für die satzungsgemäße Erfüllung der Aufgaben des Vereins. Ihm obliegen alle Aufgaben, soweit sie nicht durch Satzung der Mitgliederversammlung zugewiesen sind.

Der Vorstand ist ermächtigt, die Eintragung des Vereins in das Vereinsregister und die Anerkennung der Gemeinnützigkeit zu bewirken und das Nötige zur Aufnahme der Vereinstätigkeit zu veranlassen. Er wird ferner ermächtigt, durch Vorstandsbeschluss, der von allen Vorstandsmitgliedern zu unterzeichnen ist, die Satzung aufgrund etwaiger Beanstandungen oder Änderungs- oder Ergänzungswünsche des Registergerichts, des Finanzamtes entsprechend zu ändern.

Die Vorstandmitglieder können eine angemessene Aufwandsentschädigung bzw. eine angemessene Vergütung erhalten. Hierzu bedarf es eines Beschlusses der Mitgliederversammlung. Diese legt auch die Höhe einer Vergütung fest.

\Clause{title={Einberufung der Mitgliederversammlung}}
\label{einberufung-der-mitgliederversammlung}

Die ordentliche Mitgliederversammlung soll mindestens einmal im Kalenderjahr stattfinden. Sie wird vom Vorstand unter Einhaltung einer Frist von vier Wochen per elektronischer Übermittlung einberufen. Dabei ist die vom Vorstand festgesetzte Tagesordnung mitzuteilen.

Ort und Datum der Mitgliederversammlung sollen zudem auf der Webseite des Vereins bekannt gegeben werden.

\Clause{title={Mitgliederversammlung}}
\label{mitgliederversammlung}

In der Mitgliederversammlung hat jedes aktive Mitglied eine Stimme. Zur Ausübung des Stimmrechts kann ein anderes aktives Mitglied schriftlich bevollmächtigt werden. Die Bevollmächtigung ist für jede Mitgliederversammlung gesondert zu erteilen. Ein Mitglied darf jedoch nicht mehr als zwei fremde Stimmen vertreten.

Fördermitglieder haben in der Mitgliederversammlung ein Anwesenheits- und Antragsrecht, sind aber nicht stimmberechtigt.

Die Mitgliederversammlung ist ausschließlich für folgende Angelegenheiten zuständig:

\begin{enumerate}
\item Genehmigung des vom Vorstand aufgestellten Haushaltsplans für das nächste Geschäftsjahr; Entgegennahme des Jahresberichts des Vorstands; Entlastung des Vorstands;
\item Festsetzung der Höhe und der Fälligkeit des Jahresbeitrags;
\item Wahl und Abberufung der Mitglieder des Vorstands;
\item Beschlussfassung über Änderung der Satzung und über die Auflösung des Vereins;
\item Beschlussfassung über die Beschwerde gegen einen Ausschließungsbeschluss des Vorstands;
\item Ernennung von Ehrenmitgliedern.
\end{enumerate}

In Angelegenheiten, die in den Zuständigkeitsbereich des Vorstands fallen, kann die Mitgliederversammlung Empfehlungen an den Vorstand beschließen. Der Vorstand kann seinerseits in Angelegenheiten seines Zuständigkeitsbereichs die Meinung der Mitgliederversammlung einholen.

\Clause{title={Beschlussfassung der Mitgliederversammlung}}
\label{beschlussfassung-der-mitgliederversammlung}

Die Mitgliederversammlung wird von einem Mitglied des Vorstandes geleitet. Ist kein Vorstandsmitglied anwesend, bestimmt die Versammlung den Leiter. Bei Wahlen kann die Versammlungsleitung für die Dauer des Wahlgangs und der vorhergehenden Diskussion einem Wahlausschuss übertragen werden.

Vom Versammlungsleiter wird ein Protokollführer bestimmt. Zum Protokollführer kann auch ein Fördermitglied bestimmt werden.

Die Art der Abstimmung bestimmt der Versammlungsleiter. Die Abstimmung muss schriftlich durchgeführt werden, wenn ein Drittel der bei der Abstimmung anwesenden stimmberechtigten Mitglieder dies beantragt.

Die Mitgliederversammlung ist öffentlich.

Die Mitgliederversammlung ist beschlussfähig, wenn mindestens ein Drittel der aktiven Vereinsmitglieder anwesend ist. Bei Beschlussunfähigkeit ist der Vorstand verpflichtet, innerhalb von vier Wochen eine zweite Mitgliederversammlung mit der gleichen Tagesordnung einzuberufen. Diese ist ohne Rücksicht auf die Zahl der erschienenen Mitglieder beschlussfähig. Hierauf ist in der Einladung hinzuweisen.

Die Mitgliederversammlung fasst Beschlüsse im Allgemeinen mit einfacher Mehrheit der abgegebenen gültigen Stimmen, Stimmenthaltungen bleiben daher außer Betracht. Zur Änderung der Satzung ist jedoch eine Mehrheit von drei Viertel der abgegebenen gültigen Stimmen, zur Auflösung des Vereins eine solche von vier Fünfteln erforderlich. Eine Änderung des Zwecks des Vereins kann nur mit Zustimmung aller Mitglieder beschlossen werden. Die schriftliche Zustimmung der in der Mitgliederversammlung nicht erschienenen Mitglieder kann nur innerhalb eines Monats gegenüber dem Vorstand erklärt werden.

Für Wahlen gilt Folgendes: Hat im ersten Wahlgang kein Kandidat die Mehrheit der abgegebenen gültigen Stimmen erreicht, findet eine Stichwahl zwischen den Kandidaten statt, welche die beiden höchsten Stimmenzahlen erreicht haben.

Über die Beschlüsse der Mitgliederversammlung ist ein Protokoll aufzunehmen, das vom jeweiligen Versammlungsleiter und dem Protokollführer zu unterzeichnen ist. Es soll folgende Feststellungen enthalten: Ort und Zeit der Versammlung, die Person des Versammlungsleiters und des Protokollführers, die Zahl der erschienenen Mitglieder, die Tagesordnung, die einzelnen Abstimmungsergebnisse und die Art der Abstimmung. Bei Satzungsänderungen soll der genaue Wortlaut angegeben werden.

\Clause{title={Nachträgliche Anträge zur Tagesordnung}}
\label{nachtraegliche-antraege-zur-tagesordnung}

Jedes Mitglied kann bis spätestens eine Woche vor dem Tag der Mitgliederversammlung beim Vorstand in Textform beantragen, dass weitere Angelegenheiten nachträglich auf die Tagesordnung gesetzt werden. Der Versammlungsleiter hat zu Beginn der Mitgliederversammlung die Tagesordnung entsprechend zu ergänzen. Über Anträge auf Ergänzung der Tagesordnung, die erst in der Mitgliederversammlung gestellt werden, beschließt die Mitgliederversammlung. Zur Annahme des Antrags ist eine Mehrheit von drei Viertel der abgegebenen Stimmen erforderlich.

\Clause{title={Außerordentliche Mitgliederversammlungen}}

Der Vorstand kann jederzeit eine außerordentliche Mitgliederversammlung einberufen. Diese muss einberufen werden, wenn das Interesse des Vereins es erfordert oder wenn die Einberufung von einem Zehntel der aktiven Mitglieder oder von einem Viertel sämtlicher Mitglieder schriftlich unter Angabe des Zwecks und der Gründe vom Vorstand verlangt wird. Für die außerordentliche Mitgliederversammlung gelten die \ref{einberufung-der-mitgliederversammlung}, \ref{mitgliederversammlung}, \ref{beschlussfassung-der-mitgliederversammlung} und \ref{nachtraegliche-antraege-zur-tagesordnung} entsprechend.

\Clause{title={Auflösung des Vereins und Anfallberechtigung}}

Die Auflösung des Vereins kann nur in einer Mitgliederversammlung mit der im \ref{beschlussfassung-der-mitgliederversammlung} festgelegten Stimmenmehrheit beschlossen werden. Sofern die Mitgliederversammlung nichts anderes beschließt, sind die Vorsitzenden gemeinsam vertretungsberechtigte Liquidatoren. Die vorstehenden Vorschriften gelten entsprechend für den Fall, dass der Verein aus einem anderen Grund aufgelöst wird oder seine Rechtsfähigkeit verliert.

Bei der Auflösung des Vereins oder bei Wegfall steuerbegünstigter Zwecke fällt das Vermögen des Vereins an eine juristische Person des öffentlichen Rechts oder eine andere steuerbegünstigte Körperschaft zwecks Verwendung für die Förderung der Bildung im IT-Bereich.
\end{contract}

\end{document}
